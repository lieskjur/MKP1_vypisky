\documentclass[10pt,oneside]{article}
%\documentclass[12pt,oneside]{book}
\usepackage[utf8]{inputenc}

% Basic packages
	\usepackage{amssymb}
	\usepackage{amsmath}
	\usepackage{graphicx}
	\usepackage[czech]{babel}
	\usepackage{natbib}

% Relevant packages
	\usepackage{bm}
	\usepackage{bbm}

	%\usepackage{titling,lipsum} % titling page
	%\usepackage{nccmath} % for fleqn
	\usepackage{enumitem} %enumeration
	%\usepackage{xcolor} %colored text for cmts
	%\usepackage{bbm} % blackboard bold math
	%\usepackage{multirow} % multiple rows in tables
	%\usepackage{lscape} % for landscape section*
	%\usepackage{wrapfig}
	\usepackage{multicol}
	%\usepackage[none]{hyphenat} % no hyphenation
	%\global\hyphenpenalty=100000
	%\usepackage{caption}
	%\usepackage{subcaption} % for sub-figures
	%\usepackage{tabularx}
	%\usepackage{makecell}
	%\usepackage{nccmath}% for fleqn
	%\usepackage{float}
	%\usepackage{minted}
	%\usepackage{pdfpages}
	%\usepackage{hyperref}

% Document settings
	\usepackage{geometry}
	\geometry{
		a4paper,
		margin=15mm
	}
	\setlength{\parindent}{0pt}
	\setlength{\parskip}{0.5em}
	\setlength{\abovedisplayskip}{0em}
	\setlength{\belowdisplayskip}{0em}

	\setlist{itemsep=0em,topsep=0em}

	%\numberwithin{equation*}{section*}
	%\numberwithin{figure}{section*}

% Macros
	%\usepackage{lingebra_macros}

	\newcommand{\tenz}[1]{\vec{\vec{#1}}}

	\newcommand{\ul}[1]{\underline{#1}}
	\newcommand{\ull}[1]{\underline{\underline{#1}}}

\begin{document}
\section*{Variační princip}
\subsection*{Oblast tělesa}

Těleso reprezentujeme jako oblast $\Omega$ v euklidovském prostoru $\mathbbm{E}_3$ s hranicí $\partial\Omega$:
%
\begin{equation*}
	\partial\Omega_\sigma \cup \partial\Omega_u = \partial\Omega \ \quad \partial\Omega_\sigma \cap \partial\Omega_u = \emptyset
\end{equation*}
%
\begin{itemize}
	\item[$\partial\Omega_\sigma$] - část se silovou okrajovou podmínkou
	\item[$\partial\Omega_u$] - část s kinematickou okrajovou podmínkou
\end{itemize}

\subsection*{Celková potenciální energie}
%
\begin{equation*}
	\Pi = U - W = \frac{1}{2}\int\limits_\Omega \sigma_{ij} \varepsilon_{ij} \,dV - \Bigg( \int\limits_\Omega X_i u_i \,dV + \int\limits_{\partial\Omega_\sigma} p_i u_i \,dS \Bigg)
	\ ,\quad
	\varepsilon_{ij} = \frac{1}{2} \bigg(\frac{\partial u_i}{\partial x_j} + \frac{\partial u_j}{\partial x_i} \bigg)
\end{equation*}
%
\begin{itemize}
	\item[$U$] - deformační energie
	\item[$W$] - potenciál vnějších sil
	\item[$\tenz{\sigma}$] - tenzorové pole napjatostí
	\item[$\tenz{\varepsilon}$] - Cauchyho tenzor deformací
	\item[$\vec{X}$] - pole vnějších objemových sil
	\item[$\vec{u}$] - pole posuvů
	\item[$\vec{p}$] - vektor vnějších povrchových sil
\end{itemize}

\subsection*{Staticky přípustné pole napětí}
%
Staticky přípustné pole napětí je každé tensorové pole $\tenz{\sigma}(x,y,z)$, které splňuje rovnice rovnováhy
%
\begin{equation*}
	\frac{\partial \sigma_{ij}}{\partial x_j} + X_i = 0 \ ,\quad \sigma_{ij} \in \mathbbm{C}^1_\Omega
\end{equation*}
%
a silové okrajové podmínky:
%
\begin{equation*}
	\sigma_{ij} n^0_j = p_i \ ,\quad \forall\,[x,y,z] \in \partial\Omega_\sigma
\end{equation*}
%
kde $\vec{n}^0$ je normála na hranici $\partial\Omega$

\subsection*{Kinematicky přípustné pole posuvů}
%
Kinematicky přípustné pole posuvů je každé vektorové pole $\vec{u}(x,y,z)$, které je spojitě diferencovatelné a splňuje kinematické okrajové podmínky
%
\begin{equation*}
\vec{u}(x,y,z) = \vec{u}_{\partial\Omega}(x,y,z) \,,\quad \forall\,[x,y,z] \in \partial\Omega
\end{equation*}
%
kde $\vec{u}_{\partial\Omega}$ jsou vynucené posuvy.

\newpage
\subsection*{Princip virtuálních prací}
%
Platí identita
%
\begin{equation*}
	\int\limits_\Omega \sigma_{ij} \varepsilon_{ij} \,dV - \int\limits_\Omega X_i u_i \,dV - \int\limits_{\partial\Omega_\sigma} p_i u_i \,dS - \int\limits_{\partial\Omega_u} \sigma_{ij} n^0_j u_i \,dS = 0
	\ ,\quad
	\varepsilon_{ij} = \frac{1}{2} \bigg(\frac{\partial u_i}{\partial x_j} + \frac{\partial u_j}{\partial x_i} \bigg)
\end{equation*}
%
pokud:
\begin{itemize}
\item[$\tenz{\sigma}$] - staticky přípustné pole napjatosti
\item[$\vec{u}$] - kinematicky přípustné pole posuvů
\end{itemize}


\subsection*{Princip virtuálních posuvů}
%
Platí identita
%
\begin{equation*}
	\int\limits_\Omega \sigma_{ij}^0 \delta\varepsilon_{ij} \,dV - \int\limits_\Omega X_i \delta u_i \,dV - \int\limits_{\partial\Omega_\sigma} p_i \delta u_i \,dS = 0
	\ ,\quad
	\delta\varepsilon_{ij} = \frac{1}{2} \bigg(\frac{\partial \delta u_i}{\partial x_j} + \frac{\partial \delta u_j}{\partial x_i} \bigg)
\end{equation*}
%
pokud $\vec{u}^0 + \delta\vec{u}$ je kinematicky přípustné pole posuvů
%
\begin{equation*}
	\vec{u}^0 + \delta\vec{u} = \vec{u}_{\partial\Omega}(x,y,z) \,,\ \forall\,[x,y,z] \in \partial\Omega
	\quad \Rightarrow \quad
	\delta\vec{u} = \vec{0} \,,\ \forall\,[x,y,z] \in \partial\Omega_u
\end{equation*}
%
kde
%
\begin{itemize}[leftmargin=15mm]
	\item[$\vec{u}^0,\tenz{\sigma}^0$] - řešení úlohy pružnosti
	\item[$\delta\tenz{\varepsilon}$] - Cauchyho tenzor virtuálních deformací
	\item[$\delta\vec{u}$] - pole virtuálních posuvů %( \delta\vec{u}(\vec{r}) = \vec{0} \,, \vec{r} \in \partial\Omega_u)\\
\end{itemize}

\newpage
\section*{MKP}
%
\subsection*{1D element}
\textbf{Obecný posuv $u(\xi)$ a deformace $\varepsilon(\xi)$ elementu}
\begin{align*}
	u(\xi) &= \ul{N}^e\!(\xi)\, \ul{\delta}^e
	\,,\quad \ul{N}^e = \begin{bmatrix} 1-\frac{\xi}{l} \\ \frac{\xi}{l} \end{bmatrix} - \text{matice tvarových funkcí} \\
	\varepsilon(\xi) &= \ul{B}^e\!(\xi)\, \ul{\delta}^e
	\,,\quad \ul{B}^e = \frac{\partial \ul{N}^e}{\partial \xi} = \begin{bmatrix} -\frac{1}{l} \\ \frac{1}{l} \end{bmatrix} - \text{operátor z uzlových posuvů na deformace}
\end{align*}
kde
\begin{itemize}
	\item [$\ul{\delta}^e$] = $[\, u_1^e \, u_2^e \,]^T$ - vektor posuvů uzlových bodů
\end{itemize}

\textbf{Deformační energie elementu $U^e$}
\begin{align*}
	U^e &= \frac{1}{2} \int\limits_0^l E \varepsilon^2(\xi) A d\xi
		%= \frac{1}{2} \int\limits_0^l \varepsilon(\xi) E A\,\varepsilon(\xi) d\xi
		= \frac{1}{2}\ \ul{\delta}^{e^T} \underbrace{\int\limits_0^l \ul{B}^{e^T}\!E A\,\ul{B}^e d\xi}_{\ull{K}^e}\ \ul{\delta}^e
		= \frac{1}{2}\ \ul{\delta}^{e^T}\!\ull{K}^e\ \ul{\delta}^e
		\,,\quad
		\ull{K}^e - \text{matice tuhosti elementu}
\end{align*}

\textbf{Objemová síla $X(\xi)$ v místě $\xi$}
\begin{equation*}
	X(\xi) = \ul{N}^e\!(\xi) \ul{X}^e
	\,,\quad \ul{X}^e - \text{pole uzlových objemových sil}
\end{equation*}

\textbf{Potenciál vnější objemové síly $W^e$}
\begin{equation*}
	W^e = \int\limits_0^l u(\xi) X(\xi) A d\xi
		= \ul{\delta}^{e^T} \underbrace{\int\limits_0^l \ul{N}^{e^T} \ul{N}^e A\,d\xi\ \ul{X}^e}_{\ul{F}^e}
		= \ul{\delta}^{e^T} \ul{F}^e
		\,,\quad
		\ul{F}^e - \text{vektor ekvivalentních uzlových sil}
\end{equation*}

\textbf{Celková potenciální energie elementu}
\begin{equation*}
\Pi^e = U^e - W^e
	=  \frac{1}{2}\ \ul{\delta}^{e^T} \underbrace{\int\limits_0^l \ul{B}^{e^T}\!E A\,\ul{B}^e d\xi}_{\ull{K}^e}\ \ul{\delta}^e
	- \ul{\delta}^{e^T} \underbrace{\int\limits_0^l \ul{N}^{e^T} \ul{N}^e A\,d\xi\ \ul{X}^e}_{\ul{F}^e}
	= \frac{1}{2}\ \ul{\delta}^{e^T}\!\ull{K}^e\ \ul{\delta}^e - \ul{\delta}^{e^T} \ul{F}^e
\end{equation*}

\textbf{Celková potenciální energie modelu}
\begin{equation*}
	\begin{aligned}
		\Pi &= \sum\limits_{e=1}^{N_e} \Pi^e
			= \sum\limits_{e=1}^{N_e} \bigg(\, \frac{1}{2}\ \ul{\delta}^{e^T}\!\ull{K}^e\ \ul{\delta}^e - \ul{\delta}^{e^T} \ul{F}^e \,\bigg)
			= \sum\limits_{e=1}^{N_e} \bigg(\, \frac{1}{2}\ul{\Delta}^T \ull{\tilde{K}}^e\ \ul{\Delta} - \ul{\Delta}^T \ul{\tilde{F}}^e \,\bigg)
			= \frac{1}{2}\ul{\Delta}^T \ull{K}\ \ul{\Delta} - \ul{\Delta}^T \ul{F} 
			%= \frac{1}{2} \Delta_i^T K_{ij} \Delta_j - \Delta_i F_i
	\end{aligned}
\end{equation*}
%
kde
%
\begin{multicols}{2}
	\begin{itemize}
		\item [$\ul{\Delta}$] - vektor globálních uzlových posuvů
		\\[-.5em]
		\item [$\ull{K}$] - globální matice tuhosti
		\item [$\ull{\tilde{K}}^e$] - matice tuhosti elementu o rozměru $\ull{K}$
		\\[.5em]
		\item [$\ul{F}$] - globální vektor ekvivalentních uzlových sil
		\item [$\ul{\tilde{F}}^e$] - vektor ekvivalentních uzlových sil elementu o rozměru $\ul{F}$
	\end{itemize}
\end{multicols}

\textbf{Princip minima celkové potenciální energie}
\begin{equation*}
	\begin{aligned}
		\frac{\partial \Pi}{\partial \Delta_m} &= 0 \,;\quad m = 1,\dots,N_{DOF} \,;\quad N_{DOF} - \text{počet stupňů volnosti úlohy} \\
		&= \frac{\frac{1}{2} K_{ij} \Delta_i \Delta_j - \Delta_i F_i}{\partial \Delta_m} = \frac{1}{2} K_{mj} \Delta_j + \frac{1}{2} K_{im} \Delta_i - F_m = K_{mj} \Delta_j - F_m = 0
	\end{aligned}
\end{equation*}
%
\begin{equation*}
	\ull{K}\,\ul{\Delta} = \ul{F} \quad \Rightarrow \quad \frac{\partial \Pi}{\partial \Delta_m} = 0
\end{equation*}

\newpage
\subsection*{2D element}
\begin{multicols}{2}
\begin{itemize}
	\item [$x,y$] - souřadnice v prostoru
	\item [$\ul{\delta}^e$] - pole uzlových posuvů elementu
	\\[-.5em]
	\item [$\ul{u}$] - pole posuvů elementu
	\item [$\ul{\varepsilon}$] - pole deformací elementu
	\\[-.5em]
	\item [$\ul{\alpha}$] - Parametry pole posuvů
	\item [$\ull{A}$] - Matice koeficientů
	\item [$\ull{S}$] - Operátor parametrů $\ul{\alpha}$ na pole posuvů
	\item [$\ull{D}$] - Diferenciální operátor mezi posuvy a deformacemi
	\\[-.5em]
	\item [$\ull{N}^e$] - Matice tvarových funkcí
	\item [$\ull{B}^e$] - Operátor uzlových posunů na deformace
	\\[-.5em]
	\item [$\ul{\varepsilon}_0$] - pole deformací elementu vlivem teploty
	\item [$\ul{\varepsilon}_\sigma$] - pole deformací elementu vlivem napětí
	\\[-.5em]
	\item [$U$] - Deformační energie
	\item [$\ull{E}^e$] - matice elastických konstant 
	\item [$\ull{K}^e$] - matice tuhosti elementu
	\\[-.5em]
	\item [$\Pi$] - Celková potenciální energie
	\item [$\ul{p}$] - pole vnějších povrchových sil ???
	\item [$\ul{X}$] - pole vnějších objemových sil ???
	\\[-.5em]
	\item [$\ul{\tilde{F}}^{eE}$] - objemová síla (konstantní)
	\item [$\ul{F}^{l}$] - liniová vnější síla (abstrakce tlaku)
	%\item [$\ul{F}^{ET}$] - ekvivalentní uzlová síla objemové síly
	%\item [$W^{fe}$] - potenciál vnější objemové síly
	%\item [$W^{le}$] - potenciál liniové síly v elementu
	%\item [$\ul{\mathcal{F}}^{l}$] - vektor uzlových hodnot liniové vnější síly
\end{itemize}
\end{multicols}
%
\textbf{Pole posuvů $\ul{u}(x,y)$ a deformací $\ul{\varepsilon}(x,y)$ elementu}
%Pole posuvů $\ul{u}$ a deformací $\ul{\varepsilon}$ elementu lze vyjádřit jako funkci souřadnic $x,y$ a vektoru uzlových posuvů elementu $\ul{\delta}^e$:
%
\begin{align*}
	\ul{u}(x,y) &= \ull{A}(x,y)\,\ul{\alpha} = \ull{A}(x,y)\,\ull{S}^{-1} \ul{\delta}^e = \ull{N}^e(x,y)\,\ul{\delta}^e\\
	\ul{\varepsilon}(x,y) &= \ull{D}\,\ull{N}^e(x,y)\,\ul{\delta}^e = \ull{B}^e(x,y)\,\ul{\delta}^e
	\,,\quad
	D = \renewcommand\arraystretch{1.3}
	\begin{bmatrix}
	\frac{\partial}{\partial x} & 0 \\
	0 & \frac{\partial}{\partial y} \\
	\frac{\partial}{\partial x} & \frac{\partial}{\partial y}
	\end{bmatrix}
\end{align*}
%
%\textbf{Deformační energie}
%\begin{equation*}
%	U	= \frac{1}{2} \int\limits_\Omega \ul{\varepsilon}^T \ull{E}\ \ul{\varepsilon} \ dV
%		= \frac{1}{2}\,\ul{\delta}^{e^T}\!\Big( \int\limits_\Omega \ull{B}^T \ull{E}\,\ull{B} \ dV \Big) \ul{\delta}^e
%		= \frac{1}{2}\,\ul{\delta}^{e^T}\!\ull{K}^e\,\ul{\delta}^e
%\end{equation*}

\textbf{Změna pole deformací $\ul{\varepsilon}$ vlivem teploty}
%Izotropní tenzor deformace od teploty
%\begin{equation*}
%	\tenz{\varepsilon}_0 = \alpha \Delta T \begin{bmatrix}1&0&0\\0&1&0\\0&0&1\end{bmatrix}
%\end{equation*}
%
\begin{align*}
	\ul{\varepsilon} = \ul{\varepsilon}_\sigma + \ul{\varepsilon}_0 \,,\quad \ul{\varepsilon}_0 = \alpha \Delta T \begin{bmatrix}1\\1\\0\end{bmatrix}
\end{align*}
%
\textbf{Hustota deformační energie $\Lambda$}
\begin{equation*}
	\Lambda = \frac{1}{2}\,\ul{\varepsilon}^T \ul{\sigma}
		= \frac{1}{2}\,\ul{\varepsilon}^T \ull{E}\,\ul{\varepsilon}_\sigma
		= \frac{1}{2}\,\ul{\varepsilon}^T \ull{E}\,(\ul{\varepsilon} - \ul{\varepsilon}_0)
		= \frac{1}{2}\,\ul{\varepsilon}^T \ull{E}\,\ul{\varepsilon} - \frac{1}{2}\,\ul{\varepsilon}^T \ull{E}\,\ul{\varepsilon}_0
\end{equation*}

\textbf{Deformační energie $U$}
\begin{equation*}
U = \int\limits_\Omega \Lambda \ dV
	= \frac{1}{2} \int\limits_\Omega \ul{\varepsilon}^T \ull{E}\,\ul{\varepsilon} \ dV
	+ \frac{1}{2} \int\limits_\Omega \ul{\varepsilon}^T \ull{E}\,\ul{\varepsilon}_0 \ dV
	= \frac{1}{2}\,\ul{\delta}^{e^T}\!\underbrace{\int\limits_\Omega \ull{B}^{e^T}\!\ull{E}\ \ull{B}^e\,dV}_{\ull{K}^e} \ul{\delta}^e
	- \frac{1}{2}\,\ul{\delta}^{e^T}\!\underbrace{\int\limits_\Omega \ull{B}^{e^T}\!\ull{E}\ \ul{\varepsilon}_0 \ dV}_{\ul{F}^{ET}}
	%= \frac{1}{2}\,\ul{\delta}^e \ull{K}^e \ul{\delta}^e - \frac{1}{2}\,\ul{\delta}^e\!\int\limits_\Omega \ull{B}^{e^T}\!\ull{E}\ \ul{\varepsilon}_0 \ dV
\end{equation*}
%
%\textbf{Potenciál vnější objemové síly}
%
%\begin{equation*}
%	W^{fe} = \int\limits_\Omega \ul{u}^T(x,y) \ul{\tilde{F}}^{eE} \ dV
%		= \ul{\delta}^{e^T}\!\underbrace{\int\limits_\Omega \ull{N}^{e^T}\!(x,y)\,\ul{F}^{eE} \ dV}_{F^{V\!E}}
%\end{equation*}
\textbf{Celková potenciální energie}
\begin{equation*}
	\Pi = \frac{1}{2}\,\ul{\delta}^{e^T}\!\underbrace{\int\limits_\Omega \ull{B}^{e^T}\!\ull{E}\ \ull{B}^e\,dV}_{\ull{K}^e} \ul{\delta}^e
		- \frac{1}{2}\,\ul{\delta}^{e^T}\!\underbrace{\int\limits_\Omega \ull{B}^{e^T}\!\ull{E}\ \ul{\varepsilon}_0 \ dV}_{\ul{F}^{ET}}
		- \frac{1}{2}\,\ul{\delta}^{e^T}\!\underbrace{\int\limits_\Omega \ull{N}^{e^T}\!\ul{X} \ dV}_{\ul{F}^{ef}}
		- \frac{1}{2}\,\ul{\delta}^{e^T}\!\underbrace{\int\limits_{\partial\Omega} \ull{N}^{e^T}\!\ul{p} \ dS}_{\ul{F}^{ep}}
\end{equation*}

\newpage
\section*{Desky a skořepiny}

Kružnice s největším $R_{max}$ a nejmenším $R_{min}$ poloměrem křivosti leží v navzájem kolmých hlavních rovinách křivosti

\subsection*{Reissner-Mindlinova teorie skořepin}
%
\textbf{Předpoklad}: Hmotná normála ke střednici v nedeformovaném stavu zůstává po deformaci přímá

\subsection*{Kirchhofova teorie skořepin}
%
\textbf{Předpoklad}: Hmotná normála ke střednici v nedeformovaném stavu přejde po deformaci v normálu k deformované střednici
%\\
%\textbf{Důsledek}: Pole posuvů $\vec{u}$ je popsáno pruhybem střednice $w$
%$$\vec{u} = [u,v,w] \,,\quad w = w(x,y)$$

\subsection*{Typy deformací tenkostěnných těles}

\subsubsection*{Membránová deformace $\varepsilon_m$}
\begin{itemize}
	\item Řez střednicí se prodlouží, ale nezmění se křivost
	\item Hmotné normály zůstavají kolmé na deformovanou střednici
	\item Konstantní po tloušťce
\end{itemize}

\subsubsection*{Ohybová deformace $\varepsilon_o$}
\begin{itemize}
	\item Řez střednicí se neprodlouží, ale změní se křivost
	\item Hmotné normály zůstavají kolmé na deformovanou střednici
	\item Lineární po tloušťce
\end{itemize}

\subsubsection*{Smyková příčná deformace $\gamma_t$}
\begin{itemize}
	\item Řez střednicí se neprodlouží, nezmění se křivost
	\item Hmotné normály zůstávají přímé, ale ne obecně kolmé na deformovanou střednici
	\item Konstantní po tloušťce
\end{itemize}

\subsection*{Klasifikace skořepinových prvků}
\begin{itemize}
\item Membrány - "nemají ohybová ani příčná smyková napětí" $\varepsilon_m$
\item Desky - "nemají membránová napětí" $\varepsilon_o, \gamma_t$
\item Skořepiny - "mají ohybová napětí" $\varepsilon_m, \varepsilon_o, (\gamma_t \text{ - thick, general/universal})$
	\begin{itemize}
	\item tenké (thin) - obvykle se míní Kirchhoffovské
	\item tlusté (thick) - obvykle Mindlinovské
	\item obecné (general/universal) - obvykle mindlinovské se smykovou závorou
	\end{itemize}
\end{itemize}

\newpage
\section*{Numerická integrace}

Funkce $f(r)$ zadaná hodnotami $f_i$ na diskrétní množině bodů $r_i$ tak, že
%
\begin{equation*}
	f(r_i) = f_i \,,\quad i = 0,1,\dots,n
\end{equation*}

\subsection*{Lagrangeův interpolační polynom}
\begin{equation*}
l_j = \frac{(r-r_0)(r-r_1)\dots(r-r_{j-1})(r-r_{j+1})\dots(r-r_n)}{(r_j-r_0)(r_j-r_1)\dots(r_j-r_{j-1})(r_j-r_{j+1})\dots(r_j-r_n)}
\,;\quad
l_j(r_j) = 1 \,;\;
l_j(r_{k \neq j}) = 0 \,;\;
l_j \text{ je řádu $n$ v $r$}
\end{equation*}

\subsection*{Polynomická interpolační funkce}
\begin{equation*}
\psi(r) = f_0 l_0(r) + f_1 l_1(r) + \dots + f_n l_n(r)
\,;\quad
\psi(r_i) = f(r_i) = f_i \,;\;
\psi \in p^n
\end{equation*}

\subsection*{Newton-Cotesovo integrační schéma}
\begin{equation*}
\int\limits_a^b f(r)\;dr \approx \int\limits_a^b \psi(r)\;dr = \int\limits_a^b \sum\limits_{i=0}^n f_i l_i(r)\;dr = \sum\limits_{i=0}^n f_i \int\limits_a^b l_i(r)\;dr = (b-a) \sum\limits_{i=0}^n C_i^{\,n} f_i
\end{equation*}

\begin{center}
	\renewcommand\arraystretch{1.3}
	\begin{tabular}{|c|ccccc|}
		\hline
		$n$ & $C_0^n$ & $C_1^n$ & $C_2^n$ & $C_3^n$ & $\dots$ \\ \hline
		$1$ & $\frac{1}{2}$ & $\frac{1}{2}$ &&&\\
		$2$ & $\frac{1}{6}$ & $\frac{4}{6}$ & $\frac{1}{6}$ &&\\
		$3$ & $\frac{1}{8}$ & $\frac{3}{8}$ & $\frac{3}{8}$ & $\frac{1}{8}$ &\\ \hline
	\end{tabular}
\end{center}

\end{document}